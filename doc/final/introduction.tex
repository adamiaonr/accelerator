\section{Introduction}
\label{sec:intro}

Anonymity refers to the unlinkability of the source or\slash and destination of a communication activity, or the unlinkability of different communication activities belonging to the entity~\cite{Pfitzmann2001}. Anonymity has become an important property in the presence of censorship, privacy breaches by intelligence agencies, and pestering personalized content\slash advertisement delivery by web-based companies. Simultaneously, anonymous communication is being used for organized crime or buying\slash selling of contraband substances, such as the popular online marketplace SilkRoad. With increasing applications and user-base, a number of anonymous networks have been developed such as Mixminion\footnote{\url{http://mixminion.net/}}, Crowds\cite{Reiter:1998:CAW:290163.290168}, etc. For low-latency applications, such as web-browsing, Tor~\cite{Dingledine:2004:TSO:1251375.1251396} remains the most popular and widely-based anonymous network.


Tor's design includes a Tor client, `Onion' routers (ORs), and directory servers. All available ORs maintain live TLS connections between each other. When a Tor client wishes to connect to a server such as a website, it randomly picks three ORs from the list of available ORs and initiates Diffie-Hellman Key Exchange with each OR sequentially. Data packets are divided into cells of fixed size (called `onions') and each onion is encrypted in a layered way with the keys exchanged with each selected OR. As each OR receives the onion, it strips down the outermost layer, which is encrypted with its shared key, and forwards the remaining onion to the next OR. If an OR is the last one on the path -- known as the `exit node' -- it reconstructs the packet and forwards it to the website. Data from the website follows the reverse path. While this process only provides data confidentiality, prevention from traffic analysis is provided due to the fact that each OR forwards packets belonging to different traffic streams and therefore, the adversary cannot perform traffic analysis by observing few links/ORs in the Tor network. Additionally, Tor provides hidden services to enable a server to provide service without revealing its IP address or identity.

Despite its popularity, Tor has certain weaknesses which either have been used to breach anonymity or limits its effectiveness in other relevant applications. Unlike mix networks~\cite{Chaum:1981:UEM:358549.358563}, Tor does not perform any explicit buffering, delaying, or reordering of onions. This means that in the absence of sufficient traffic the traffic patterns created at source travel deep into the tor network without significant modifications and can be identified there. Murdoch and Danezis~\cite{Murdoch:2005:LTA:1058433.1059390} used this to launch a side channel attack that reveals the ORs used by a certain anonymous traffic stream. Since, Tor uses the same path for multiple traffic streams belonging to the same client in order to avoid path setup delay, this attack can be used to correlate two communication activities of the same client breaching its anonymity. In spite of optimization done to minimize packet delay, cryptographic processing of numerous packets leads still induces packet delay of the order some milliseconds which makes Tor unsuitable to other applications that have stringent  delay requirements, such as VoIP and streaming. 

Certain solutions to reduce the packet or path setup delay in Tor have been proposed. LASTor~\cite{6679287} is a modification in the client side of Tor, which ensures that OR chosen for a path are not separated demographically so ensure low transmission delay. Christin et al.~\cite{Christin2013} proposed to reuse the path chosen by a hidden service provider to reach hidden service's directory server for the rendezvous point for the server and the client. While these solutions indeed provide reduction in packet/path setup delay they require fundamental change in Tor's design and have  implications on the anonymity guarantees of Tor. In this work, we intend to reduce overall delay of Tor with the use of Intel's DPDK packet processing framework. (...) DPDK uses optimized functions of handling batches of data and uses specialized data structures, such as ring buffers, to achieve significant improvement in packet processing delay. This project explores benefits of DPDK translate to better performance for Tor and the impact of this approach on Tor's anonymity guarantees. 
