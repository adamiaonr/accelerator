\section{Related Work}
\label{sec:re-work}
Tor is a widely used anonymous network on the Internet. There is a large body of work which have attempted to improve Tor’s performance. Some approaches include the refinement of Tor’s relay selection strategy \cite{6679287, Sherr10a3:an, Snader08atune-up}. Akhoondi et al.~\cite{6679287}have proposed an approach that brings performance gains by trading off between the anonymity and latency. Such approaches are orthogonal to our approach and can be used in conjunction with acceleraTor as we do not modify the path selection algorithm.
Jansen et al. ~\cite{184431} propose a socket management algorithm that uses real-time kernal information to compute the amount of information to write to each socket to avoid congestion bottleneck which is claimed to occur in egress kernel socket.
Torchestra ~\cite{Gopal:2012:TRI:2381966.2381972} uses TCP connections to carry bulk traffic in order to isolate effects of congestions betwen different traffic classes and their proposal is based on the analysis that bulk downloads cause delays for interactive traffic. 
AlSabah et al. ~\cite{AlSabah:2011:DTO:2032162.2032170} proposed DefenestraTor which modifies traffic management in Tor relays to reduce queuing delays. Panchenko et al. ~\cite{5230652} proposed path selection algorithms to choose the path inferring from available bandwidth on each relay.  
As described, many works have focused on the choosing of path selection algorithms but authors ~\cite{Panchenko:2008:PAA:1371602.1371906} have also studied the tradeoff between improved performance and anonymity. However, our approach does not deal with path selection algorithms, thus will not hurt anonimity due to having low diversity in path selection. Other approaches to improve performance of Tor have proposed  incentive-based means to users to relay Tor circuits ~\cite{Jansen:2010:RNT:1866307.1866344, Jansen_lira:lightweight}. Nowlan et al. ~\cite{179191} proposed the use of uTCP and uTLS ~\cite{180706} to tackle the head-of-line blocking problem between Tor cirtuis sharing TCP connections to relax the in-order delivery assumptions of TCP. 