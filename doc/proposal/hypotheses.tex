\section{Hypotheses}
\label{sec:hypotheses}

We expect that AcceleraTOR increases the performance of Tor networks by benefiting from the use of DPDK for the ToR network. The key elements in DPDK (buffer, queue, packet flow classification and NIC management) and select use of these libraries are expected to incur low latency compared to the status quo.

AcceleraTOR hopes to achieve low latency by modifying the ways packets are being handled in ORs. The question lies in the percentage of gain that DPDK can bring to the Tor network which can be roughly answered by first pinpointing the main sources in high latency in ToR network. In~\cite{Dingledine:2004:TSO:1251375.1251396} Dingledine et al. state that the `delay' incurred by Tor is mostly due to (1) the establishment of Tor virtual circuits among Onion Routers (ORs) (``tenths of seconds''); (2) rotating between circuits to limit linkability among multiple TCP streams established by some user. In addition, the operations which mostly contribute to circuit establishment delay are (1) public key encryption; and (2) network latency. These suggest that the bottleneck of the network does not lie in user-level packet forwarding operations, and therefore not influenced by the DPDK. Therefore, the enhancements brought about by DPDK may not be \textit{too} significant: nevertheless, we deem it worthwhile to see the implication of introducing DPDK into Tor network and see further implication on the impact on Tor anonymity. Especially, we can infer some impact on anonymity, if any, by the batching of packets used as an optimization technique by DPDK. 

There have been several efforts trying to reduce the latency in Tor networks. Nowlan et al.~\cite{179191} propose latency reduction by relaxing the requirements of strict in-order delivery of TCP. The chosen method is comparable to our proposed system by modifying the ways that packets are forwarded at the application level. The result claims to have alleviated application-perceived latency. Our approach anticipates the performance benefit by introducing DPDK - which bring the performance gain through buffer, memory, packet flow, and NIC management. Akhoondi et al. propose LasTor~\cite{Akhoondi:2012:LLA:2310656.2310712} as an attempt to reduce the latency in ToR network by making the client-side modifications and not requiring the revamp of the ToR architecture. However, this performance gain is at the expense of loss of anonymity. Our proposed  approach is not expected to bring anonymity trade-off like LasTor. Mainly due to orthogonal approaches being taken by LasTor and AcceleraTOR, both approaches can be coupled to bring even further performance gain in ToR networks. 
