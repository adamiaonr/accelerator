\section{Experiments}
\label{sec:experiments}

\textbf{Experiments:}

\begin{itemize}
	
	\item Setup a virtualized network with several Tor ORs. We propose a simple setup (i.e. 5-10 nodes), appropriate to highlight the differences between the proposed system --- acceleraTor --- and one or more non-DPDK-enhanced Tor variant(s).

	\item Besides comparing the clockWatcher system against a `vanilla' version of Tor, we propose to compare it against recently proposed `latency-reduction' approaches, such as uTor~\cite{179191} or LasTor~\cite{Akhoondi:2012:LLA:2310656.2310712}

	\item To take advantage of the fast packet processing capabilities provided by DPDK, we shall use NICs supported by the framework (e.g. see DPDK: Supported NICs)

	\item We may need to deal with the difficulty in `measuring' the level of anonymity provided by acceleraTor. 

\end{itemize}

We will mainly focus on measuring latency and data-loss for different sets of applications\slash operations:

\begin{itemize}

	\item Web content download (e.g. web pages, video ‘pseudo-streaming’ over HTTP, etc.)
	\item Time sensitive network applications such as live video streaming, VoIP applications, video chat, etc.

\end{itemize}

% Choose a set of applications for the demo / performance analysis (ZJ)
% High definition video streaming (e.g. 720p, 1080p, 4K)
% Measure metric: the time takes to buffer for 1 minute of the same video
% Quality of voice streaming such quality of Skype
% Measure metric: number of pauses over a period of time for the same recorded voice clip
% Website downloading speed
% Measure metric: the time taken to load the same website
% Notes: seems that most of the services not usable or having additional inconvenience over Tor are not due to speed but two reasons (YOU’RE RIGHT ON THIS ONE)
% the website does not accept anonymous user, such as Yelp which does not allow Tor users to access website even after keying reCaptcha
% Browser support: mainly lack of plugins and also Torbutton disables many types of active content such as Video conferencing
