\section{Proposal}
\label{sec:proposal}

Tor provides increased anonymity in network communication by the use of set of relays. Tor networks introduce the idea of taking a random pathway through set of relays for the purpose of hiding the origin of data packets. Introduction of these extra `random' hops is a fundamental idea in increasing anonymity and also the cause of increased latency.
 
The goal of our proposed system, AcceleraTOR, is to enhance the performance of Tor by reducing observed latency in streams and loss in data segment events (observed in many download events). External research performed in Dec 2010 to Jan 2011 found that ``(...) the average ratio of HTTP request durations (Tor/direct) [from around the world] is 4.1 (...)''~\cite{fi4020488}. However, recent research suggests the comparability between download speed between the direct Internet and ToR browsing\footnote{\label{note1}\url{http://tor.stackexchange.com/questions/4459/how-does-tor-network-latency-speed-compare-with-direct-internet}}. Download speeds alone cannot quantify the comparability alone and other metrics such as packet loss and performance over time-sensitive user-experience. The objective then would be to see whether the merging Tor with DPDK can bring performance benefits, thus increase user-experience. A vast majority of users - accounting for over 90\% of TCP connections on Tor~\cite{Mccoy:2008:SLD:1428259.1428264} - is known to use Tor for interactive traffic. Such user pattern implies a bigger impact of decreasing latency in ToR network. The performance gain in Tor may benefit applications such as VPN, VoIP, and\slash or video streaming especially ones with strict timing requirements. 

Our approach to achieve this goal consists of reducing the latency incurred by Tor Onion Routers (ORs) in processing and relaying packets at the application level. In order to achieve our end goal, mainly performance gain in ToR, we will explore the followings:
We will make use of the Data Plane Development Kit (DPDK)\footnote{\url{http://dpdk.org}}, which allows for fast data plane performance (packet processing) at a user-level. In addition, we will investigate:

\begin{itemize}
	\item Integrate recently proposed techniques for latency reduction in Tor with DPDK’s Environment Abstraction Layer (EAL) (e.g. third-party fast path user-level stacks such as uTor~\cite{179191}). We expect that performance of ToR will benefit from DPDK optimizations to enhance the performance of existing implementations.
	\item Introduce novel ideas - not previously introduced by others - to improve Tor's performance.
\end{itemize}
